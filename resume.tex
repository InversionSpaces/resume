% neomake: skip
\documentclass{article}

\usepackage{cmap}
\usepackage{array}
\usepackage{enumitem}
\usepackage{makecell}
\usepackage[T2A]{fontenc}
\usepackage[utf8]{inputenc}
\usepackage[english,russian]{babel}
\usepackage{indentfirst}
\usepackage{amssymb}
\usepackage{amsmath}
\usepackage{multicol}
\usepackage{fontawesome}
\usepackage{verbatim}

\usepackage{geometry}
\geometry{top=10mm}
\geometry{bottom=10mm}
\geometry{left=15mm}
\geometry{right=15mm}

\pagenumbering{gobble}

\usepackage[dvipsnames]{xcolor}
\usepackage[colorlinks  = true,
            linkcolor   = blue,
            urlcolor    = blue,
            citecolor   = black,
            anchorcolor = black]{hyperref}

\renewcommand{\maketitle}{
    \Huge
    \begin{center}
        \textbf{Воробьев Михаил}
    \end{center}

    \small
    \begin{center}
    \faMobile \hspace{0.1cm} $\boldsymbol{+}$7(977)938-28-68 
    \hfill
    \faEnvelope \hspace{0.1cm} \href{mailto:vorobev.mk@phystech.edu}{vorobev.mk@phystech.edu}
    \hfill
    \faPaperPlane \hspace{0.1cm} \href{https://t.me/purely_injected}{@purely\_injected}
    \hfill
    \faGithub \hspace{0.1cm} \href{https://github.com/InversionSpaces}{InversionSpaces}
    \end{center}
}

\usepackage{titlesec}
\titleformat{\section}{\Large\bf\raggedright}{}{0.5em}{}[{\titlerule[1pt]}]
\titlespacing{\section}{0pt}{3pt}{7pt}
\titleformat{\subsection}{\large\bfseries\raggedright}{}{0em}{\underline}%[\rule{3cm}{.2pt}]
\titlespacing{\subsection}{0pt}{7pt}{7pt}


\newcommand{\entry}[3]{
	\begin{tabular}{ c | c }
    \begin{minipage}{0.05\linewidth}
    	\begin{center}
    		#1
    	\end{center}
    \end{minipage} 
    &
    \begin{minipage}{0.85\linewidth}
        \textbf{#2} \\ \footnotesize{#3}
    \end{minipage}
    \end{tabular}
}

\newcommand{\interval}[2]{
	#1 \\ $\downarrow$ \\ #2
}


\setlist[itemize]{noitemsep, topsep=0pt}

\begin{document}
    \maketitle
    \small
    
    \section{Образование}
        \entry {\interval{2019}{н.в.}}
        {Московский физико-технический институт - \\
         Физтех-школа прикладной математики и информатики - \\
         Информатика и вычислительная техника}
        {Очная форма обучения, бакалавриат.}

    \section{Работа и волонтерство}
        \entry {\interval{Фев. 2021}{н.в.}}
        {Тинькофф - Отдел кредитных и брокерских систем - Младший специалист}
        {Бэкенд разработка на Scala: 
        	\begin{itemize}
        		\item Разработка нового и поддержание старого функционала нескольких внутренних микросервисов.
        		\item Разработка с применением брокеров сообщений: RabbitMQ и Kafka.
        		\item Разработка с применением баз данных: MongoDB и PostgreSQL.
        		\item Разработка бизнес функционала бэкенда кредитного онлайн брокера.
        	\end{itemize}
        }
        
        \vspace{.1cm}
        
        \entry {Июль 2021}
        {Летняя школа Слон в Пущино - Волонтер-преподаватель}
        {Проведение курсов:
        	\begin{itemize}
        		\item Формальные языки.
        		\item Математическая логика.
        	\end{itemize}
    	}  

    \section{Курсы}
    \entry {2021
        }
        {МФТИ - Теория и практика многопоточной синхронизации }
        { Язык C\texttt{++}. Реализация библиотеки для работы с многопоточностью: Fibers, Futures, ThreadPools.  }
    
    \vspace{.1cm}
    
    \entry {2021
        }
        {МФТИ - Распределенные системы }
        { Язык C\texttt{++}. Реализация алгоритмов распределенных консенсуса и репликации: Paxos, MultiPaxos, Raft.  }
    
    \vspace{.1cm}
    
    \entry {2020
       %\href {https://github.com/InversionSpaces/problems} {Репозиторий} 
        }
        {Тинькофф Финтех - Курс разработки на Scala }
        { Язык Scala. Разработка RESTful сервиса StopLoss-TakeProfit для тинькофф инвестиций. }
    
    \vspace{.1cm}
    
    \entry {\interval{2019}{2020}
   %\href {https://github.com/InversionSpaces/problems} {Репозиторий} 
    }
    {Технотрек Mail.ru Group - \\
    Курс "Введение в промышленное программирование и структуры данных"}
    { Языки C, C\texttt{++}. Разработка эмулятора стековой машины и компилятора высокоуровнего языка. }
    
    \vspace{.1cm}  

    \entry {\interval{2017}{2019}
   %\href {https://github.com/Gravifarsh/FreyaCode} {Репозиторий} 
    }
    {Воздушно-инженерная школа Cansat в России - \\
    Разработка спускаемого аппарата}
    { Язык C для AVR и STM32. Ведущий разработчик в команде. \\ Единственный разработчик в год победы команды в регулярной лиге. } 
    
    \vspace{.1cm}
     
    \section{Проекты}
        
    \entry {2020}
    {\href {https://gitlab.com/Inversion/gainy}{gainy} - Сервис с HTTP RESTful API}
    { Язык Scala. Cервис StopLoss-TakeProfit, интегрированный с тинькофф.инвестиции. } 
    
    \vspace{.1cm}
        
    \entry {2019}
    {\href {https://github.com/InversionSpaces/PEGgen}{PEGgen} - Утилита комнадной строки}
    { Язык Python3. Генерация кода парсера на рекурсивном спуске из формальной грамматики.} 
    
    \vspace{.1cm}
    
    \entry {2021}
    {\href {https://gitlab.com/Inversion/yaelc} {yaelc} - Компилятор java-подобного языка}
    { Язык C\texttt{++}. Реализация сканера (GNU Flex), LR-парсера (GNU Bison), AST, таблицы типов. } 
    
    \vspace{.1cm}
    
    \entry {\interval{2017}{2019}}
    {\href {https://github.com/cansat-rsce/librscs} {librscs} - Библиотека драйверов перифирии для микроконтроллеров AVR}
    { Язык C для AVR. Реализация драйвера модема Iridium9602 и тестирование других модулей. } 

    \section{Навыки}
    	\begin{tabular}{ >{\bfseries}r | l }
    		Языки программирования & Scala, C/C\texttt{++}, Python3 \\
    		Технологии & MongoDB, PostgreSQL, RabbitMQ, Kafka  \\
    		Фреймворки и библиотеки & ZIO, cats, fs2, akka-http \\
    		Инструменты & git, sbt, Gitlab CI, liquibase \\
    		Языки & Русский(родной), English(intermediate)
    	\end{tabular} 
        
    \vspace{\fill}
    \begin{center}
        \large
        \href {https://github.com/InversionSpaces/resume}{Актуальная версия этого резюме}
    \end{center}
\end{document}
