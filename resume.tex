% neomake: skip
\documentclass{article}

\usepackage{cmap}
\usepackage[T2A]{fontenc}
\usepackage[utf8]{inputenc}
\usepackage[english,russian]{babel}
\usepackage{indentfirst}
\usepackage{amssymb}
\usepackage{amsmath}
\usepackage{multicol}
\usepackage{fontawesome}

\usepackage{geometry}
\geometry{top=20mm}
\geometry{bottom=10mm}
\geometry{left=20mm}
\geometry{right=10mm}

\pagenumbering{gobble}

\usepackage[dvipsnames]{xcolor}
\usepackage[colorlinks  = true,
            linkcolor   = blue,
            urlcolor    = blue,
            citecolor   = black,
            anchorcolor = black]{hyperref}

\renewcommand{\maketitle}{
    \Huge
    \begin{center}
        \textbf{Воробьев Михаил}
    \end{center}

    \large
    \begin{minipage}{0.4\textwidth}
        \faMobile \hspace{0.1cm} $\boldsymbol{+}$7(977)938-28-68\\[0.3em]
        \faEnvelope \hspace{0.1cm} \href{mailto:vorobev.mk@phystech.edu}{vorobev.mk@phystech.edu}
    \end{minipage}
    \hfill
    \begin{minipage}{0.4\textwidth}
        \begin{flushright}
            \faPaperPlane \hspace{0.1cm} \href{https://t.me/purely_injected}{@purely\_injected}\\[0.3em]
            \faGithub \hspace{0.1cm} \href{https://github.com/InversionSpaces}{InversionSpaces}
        \end{flushright}
    \end{minipage}
}

\usepackage{titlesec}
\titleformat{\section}{\Large\bf\raggedright}{}{0.5em}{}[{\titlerule[1pt]}]
\titlespacing{\section}{0pt}{3pt}{7pt}
\titleformat{\subsection}{\large\bfseries\raggedright}{}{0em}{\underline}%[\rule{3cm}{.2pt}]
\titlespacing{\subsection}{0pt}{7pt}{7pt}


\newcommand{\entry}[3]{
    \begin{minipage}[t]{.11\linewidth}
        \hfill \textsc{#1}
    \end{minipage}
    \hfill\vline\hfill
    \begin{minipage}[t]{.80\linewidth}
        \textbf{#2}\\
        \footnotesize{#3}
    \end{minipage}
}

\begin{document}
    \maketitle
    \small
    
    \section{Образование}
        \entry {2019 -- 2023}
        {Московский физико-технический институт\\
         Физтех-школа прикладной математики и информатики\\
         Информатика и вычислительная техника}
        {Очная форма обучения, бакалавриат, 2 курс}


    \section{Курсы и конкурсы}
        \entry {2019}
        {Технотрек Mail.ru Group\\
        Введение в промышленное программирование и структуры данных}
        {Язык C}

        \vspace{.2cm}

        \entry {2017 -- 2019}
        {Воздушно-инженерная школа Cansat в России\\
        Разработка бортового кода для спускаемого аппарата}
        {Язык C для AVR и STM32}        

    \section{Проекты}
    \entry {2019}
    {PEGgen - генератор C++ кода для парсера, применяющего рекурсивный спуск\\
    На основе контекстно-свободной грамматики в BNF
    }
    {Язык Python3} 
    
    \vspace{.2cm}
    
    \entry {2020}
    {Сайт Физтех-школы прикладной математики и информатики\\
    Разработка Frontend}
    {Язык JS и React}

    \section{Навыки}
        \begin{multicols}{2}
            \subsection{Языки Программирования}
                \begin{itemize}
                    \item C/C$\boldsymbol{++}$
                    \item Python3
                    \item x86 Assembly
                    \item JavaScript
                    \item Scheme
                \end{itemize}

            \subsection{Операционные системы}
               \begin{itemize}
                    \item GNU/Linux (debian-based, arch-based)
               \end{itemize}
            \subsection{Языки}
                \begin{itemize}
                    \item Русский - родной
                    \item English - Intermediate
                \end{itemize}
            \subsection{Инструменты}
                \begin{itemize}
                    \item git
                    \item docker
                    \item \LaTeX
                \end{itemize}
        \end{multicols}

    \section{Другое}
        \begin{itemize}
            \item Люблю передавать свои знания при возможности - объяснять или учить
            \item Люблю гулять и кататься на велосипеде
            \item Люблю тайлинговые оконные менеджеры
        \end{itemize}
        
    \vspace{\fill}
    \begin{center}
        \large
        \href {https://github.com/InversionSpaces/resume}{Актуальная версия этого резюме}
    \end{center}
\end{document}
