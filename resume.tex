% neomake: skip
\documentclass{article}

\usepackage{cmap}
\usepackage{array}
\usepackage{enumitem}
\usepackage{makecell}
\usepackage[T2A]{fontenc}
\usepackage[utf8]{inputenc}
\usepackage[english,russian]{babel}
\usepackage{indentfirst}
\usepackage{amssymb}
\usepackage{amsmath}
\usepackage{multicol}
\usepackage{fontawesome}
\usepackage{verbatim}

\usepackage{geometry}
\geometry{top=10mm}
\geometry{bottom=10mm}
\geometry{left=15mm}
\geometry{right=15mm}

\pagenumbering{gobble}

\usepackage[dvipsnames]{xcolor}
\usepackage[colorlinks  = true,
            linkcolor   = blue,
            urlcolor    = blue,
            citecolor   = black,
            anchorcolor = black]{hyperref}
            
\newif\ifen
\newif\ifru

\newcommand{\en}[1]{\ifen#1\fi}
\newcommand{\ru}[1]{\ifru#1\fi}

%\rutrue
\entrue

\renewcommand{\maketitle}{
    \Huge
    \begin{center}
        \textbf{
        \ru{Воробьев Михаил}
        \en{Vorobev Mikhail}
        }
    \end{center}

    \small
    \begin{center}
    \faMobile \hspace{0.1cm} $\boldsymbol{+}$7(977)938-28-68 
    \hfill
    \faEnvelope \hspace{0.1cm} \href{mailto:vorobev.mk@phystech.edu}{vorobev.mk@phystech.edu}
    \hfill
    \faPaperPlane \hspace{0.1cm} \href{https://t.me/purely_injected}{@purely\_injected}
    \hfill
    \faGithub \hspace{0.1cm} \href{https://github.com/InversionSpaces}{InversionSpaces}
    \end{center}
}

\usepackage{titlesec}
\titleformat{\section}{\Large\bf\raggedright}{}{0.5em}{}[{\titlerule[1pt]}]
\titlespacing{\section}{0pt}{3pt}{7pt}
\titleformat{\subsection}{\large\bfseries\raggedright}{}{0em}{\underline}%[\rule{3cm}{.2pt}]
\titlespacing{\subsection}{0pt}{7pt}{7pt}


\newcommand{\entry}[3]{
	\begin{tabular}{ c | c }
    \begin{minipage}{0.05\linewidth}
    	\begin{center}
    		#1
    	\end{center}
    \end{minipage} 
    &
    \begin{minipage}{0.85\linewidth}
        \textbf{#2} \\ \footnotesize{#3}
    \end{minipage}
    \end{tabular}
}

\newcommand{\interval}[2]{
	#1 \\ $\downarrow$ \\ #2
}


\setlist[itemize]{noitemsep, topsep=0pt}

\begin{document}
    \maketitle
    \small
    
    \section{\ru{Образование}\en{Education}}
        \entry {\interval{2019}{\ru{2022}\en{2022}}}
        {\ru{Московский физико-технический институт}\en{Moscow Institute of Physics and Technology} - \\
       	\ru{Физтех-школа прикладной математики и информатики}\en{Phystech School of Applied Mathematics and Informatics} - \\
       	\ru{Информатика и вычислительная техника}\en{Engineering and Computer Science}
       	}
        {\ru{Очная форма обучения, бакалавриат, неоконченное (2 курса).}\en{Bachelor's degree, unfinished (2 years).}}

    \section{\ru{Работа и волонтерство}\en{Work and teaching experience}}
        \entry {\interval{\ru{Фев.}\en{Feb.} 2021}{\ru{Июль}\en{July} 2022}}
        {\ru{Тинькофф}\en{Tinkoff} - \ru{Отдел Кредитных и Брокерских Систем}\en{Credit and Brokerage Systems} - Backend \ru{разработчик}\en{Engineer}}
        {\ru{Бэкенд разработка на Scala}\en{Backend Scala development}: 
        	\begin{itemize}
        		\item \ru{Разработка нового и поддержание старого функционала нескольких внутренних микросервисов}\en{Maintenance and development of several in-house microservices}.
        		\item \ru{Разработка с применением брокеров сообщений: RabbitMQ и Kafka}\en{Integration of message brokers: RabbitMQ and Kafka}.
        		\item \ru{Разработка с применением баз данных: MongoDB и PostgreSQL}\en{Integration of databases: MongoDB and PostgreSQL}.
        		\item \ru{Разработка бизнес функционала бэкенда кредитного онлайн брокера}\en{Implementation of several significant buisness features in online credit broker}.
        	\end{itemize}
        }
        
        \vspace{.1cm}
        
        \entry {\interval{\ru{Авг.}\en{Aug.} 2022}{\ru{н.в.}\en{today}}}
        {Mosaic Research - Crypto HFT - Backend \ru{разработчик}\en{Engineer}}
        {\ru{Бэкенд разработка на C\texttt{++}}\en{Backend C\texttt{++} development}: 
        	\begin{itemize}
        		\item \ru{Разработка и поддержка инфраструктуры для торговли криптовалютами} \en{Development and maintenance of infrastructure for crypto trading}.
        		\item \ru{Создание}\en{Setting up} CI pipeline \ru{для}\en{for} CMake projects: building, testing, caching, automatic dependecies updating.
        		\item \ru{Имплементация протокола коммуникации между компонентами с использованием} \en{Implementing microservices communication protocol with} Aeron.
        	\end{itemize}
        }
        
        \vspace{.1cm}
        
        \entry {2021, 2022}
        {\ru{Летняя Школа Слон в Пущино}\en{Summer School <<Slon>> in Pushchino} - \ru{Волонтер-Преподаватель}\en{Volunteer-Tutor}}
        {\ru{Проведение курсов}\en{Courses taught}: \ru{Формальные языки}\en{Formal languages}, \ru{Математическая логика}\en{Mathematical logic}, \ru{ФП в }\en{FP in} Scala, Python \ru{с нуля}\en{from scratch}.
    	}  

    \section{\ru{Курсы}\en{Courses}}
    \entry {2021
        }
        {\ru{МФТИ}\en{MIPT} - \ru{Теория и практика многопоточной синхронизации}\en{Concurrency course} }
        {\ru{Язык C\texttt{++}}\en{C\texttt{++}}. \ru{Реализация библиотеки для работы с многопоточностью}\en{Implementation of simple concurrency library}: fibers, futures, thread pool.  }
    
    \vspace{.1cm}
    
    \entry {2021
        }
        {\ru{МФТИ}\en{MIPT} - \ru{Распределенные системы}\en{Distributed systems} }
        {\ru{Язык C\texttt{++}}\en{C\texttt{++}}. \ru{Реализация алгоритмов распределенных консенсуса и репликации}\en{Implementation of distributed consensus and replication algorithms}: paxos, multi paxos, raft.  }
    
    \vspace{.1cm}
    
    \entry {2020
       %\href {https://github.com/InversionSpaces/problems} {Репозиторий} 
        }
        {\ru{Тинькофф.Финтех}\en{Tinkoff.Fintech} - \ru{Курс разработки на Scala}\en{Scala development course} }
        { \ru{Язык Scala}\en{Scala}. \ru{Разработка RESTful сервиса StopLoss-TakeProfit для тинькофф инвестиций}\en{Impementation of StopLoss-TakeProfit service integrated with tinkoff.investments}. }
    
    \vspace{.1cm}
    
    \entry {\interval{2019}{2020}
   %\href {https://github.com/InversionSpaces/problems} {Репозиторий} 
    }
    {Mail.ru \ru{Технотрек} - \ru{Введение в промышленное программирование и структуры данных}\en{Introduction to production programming and data structures}}
    { \ru{Языки C, C\texttt{++}}\en{C, C\texttt{++}}. \ru{Разработка эмулятора стековой машины и компилятора высокоуровнего языка для нее}\en{Implementation of stack machine emulator and high-level language compiler for it}.} 
    
    \vspace{.1cm}  

    \entry {\interval{2017}{2019}
   %\href {https://github.com/Gravifarsh/FreyaCode} {Репозиторий} 
    }
    {\ru{Воздушно-инженерная школа CanSat в России}\en{Air engineering school in Russia} - \ru{Разработка спускаемого аппарата}\en{Sounding rocket payload development}}
    { \ru{Язык C для AVR и STM32}\en{C for AVR and STM32}. \ru{Разработка бортовой прошивки спускаемого аппарата}\en{Development of on-board firmware and peripheral drivers}. } 
    
    \vspace{.1cm}
     
    \section{\ru{Проекты}\en{Projects}}
        
    \entry {2020}
    {\href {https://gitlab.com/Inversion/gainy}{gainy} - \ru{Сервис с REST API}\en{Service with REST API}}
    { \ru{Язык Scala}\en{Scala}. \ru{Cервис StopLoss-TakeProfit, интегрированный с тинькофф.инвестиции}\en{StopLoss-TakeProfit service integrated with tinkoff.investments}. } 
    
    \vspace{.1cm}
        
    \entry {2019}
    {\href {https://github.com/InversionSpaces/PEGgen}{PEGgen} - \ru{Утилита комнадной строки}\en{Command line utility}}
    { \ru{Язык Python3}\en{Python3}. \ru{Генерация кода парсера на рекурсивном спуске из формальной грамматики}\en{Generation of recursive descent parser from formal grammar}.} 
    
    \vspace{.1cm}
    
    \entry {2021}
    {\href {https://gitlab.com/Inversion/yaelc} {yaelc} - \ru{Компилятор java-подобного языка}\en{Compiler of java-like language}}
    { \ru{Язык C\texttt{++}}\en{C\texttt{++}}. \ru{Сканер (GNU Flex), LR-парсер (GNU Bison), AST, таблица типов}\en{Scanner (GNU Flex), LR-parser (GNU Bison), AST, type table}.} 
    
    \vspace{.1cm}
    
    \entry {\interval{2017}{2019}}
    {\href {https://github.com/cansat-rsce/librscs} {librscs} - \ru{Библиотека драйверов перифирии для микроконтроллеров AVR}\en{Peripheral drivers library for AVR}}
    { \ru{Язык C для AVR}\en{C for AVR}. \ru{Драйвер модема Iridium9602 и тестирование других модулей}\en{Driver for Iridium9602 modem and testing of other modules}. } 

    \section{\ru{Навыки}\en{Skills}}
    	\begin{tabular}{ >{\bfseries}r | l }
    		\ru{Языки программирования}\en{Programming languages} & Scala, C/C\texttt{++}, Python3 \\
    		\ru{Технологии}\en{Technologies} & MongoDB, PostgreSQL, RabbitMQ, Kafka  \\
    		\ru{Фреймворки и библиотеки}\en{Frameworks and libraries} & ZIO, cats, fs2, akka-http, Aeron \\
    		\ru{Инструменты}\en{Tools} & git, cmake, sbt, Gitlab CI, Github Actions, liquibase \\
    		\ru{Языки}\en{Languages} & Русский(родной), English(B2)
    	\end{tabular} 
        
    \vspace{\fill}
    \begin{center}
        \large
        \href {https://github.com/InversionSpaces/resume}{\ru{Актуальная версия этого резюме}\en{Up to date version of this CV}}
    \end{center}
\end{document}
