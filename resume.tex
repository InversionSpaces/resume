% neomake: skip
\documentclass{article}

\usepackage{cmap}
\usepackage[T2A]{fontenc}
\usepackage[utf8]{inputenc}
\usepackage[english,russian]{babel}
\usepackage{indentfirst}
\usepackage{amssymb}
\usepackage{amsmath}
\usepackage{multicol}
\usepackage{fontawesome}

\usepackage{geometry}
\geometry{top=20mm}
\geometry{bottom=10mm}
\geometry{left=20mm}
\geometry{right=10mm}

\pagenumbering{gobble}

\usepackage[dvipsnames]{xcolor}
\usepackage[colorlinks  = true,
            linkcolor   = blue,
            urlcolor    = blue,
            citecolor   = black,
            anchorcolor = black]{hyperref}

\renewcommand{\maketitle}{
    \Huge
    \begin{center}
        \textbf{Воробьев Михаил}
    \end{center}

    \small
    \begin{center}
    \faMobile \hspace{0.1cm} $\boldsymbol{+}$7(977)938-28-68 
    \hfill
    \faEnvelope \hspace{0.1cm} \href{mailto:vorobev.mk@phystech.edu}{vorobev.mk@phystech.edu}
    \hfill
    \faPaperPlane \hspace{0.1cm} \href{https://t.me/purely_injected}{@purely\_injected}
    \hfill
    \faGithub \hspace{0.1cm} \href{https://github.com/InversionSpaces}{InversionSpaces}
    \end{center}
    %\begin{minipage}{0.4\textwidth}
    %    \faMobile \hspace{0.1cm} $\boldsymbol{+}$7(977)938-28-68\\[0.3em]
    %    \faEnvelope \hspace{0.1cm} \href{mailto:vorobev.mk@phystech.edu}{vorobev.mk@phystech.edu}
    %\end{minipage}
    %\hfill
    %\begin{minipage}{0.4\textwidth}
    %    \begin{flushright}
    %        \faPaperPlane \hspace{0.1cm} \href{https://t.me/purely_injected}{@purely\_injected}\\[0.3em]
    %        \faGithub \hspace{0.1cm} \href{https://github.com/InversionSpaces}{InversionSpaces}
    %    \end{flushright}
    %\end{minipage}
}

\usepackage{titlesec}
\titleformat{\section}{\Large\bf\raggedright}{}{0.5em}{}[{\titlerule[1pt]}]
\titlespacing{\section}{0pt}{3pt}{7pt}
\titleformat{\subsection}{\large\bfseries\raggedright}{}{0em}{\underline}%[\rule{3cm}{.2pt}]
\titlespacing{\subsection}{0pt}{7pt}{7pt}


\newcommand{\entry}[3]{
    \begin{minipage}[t]{.11\linewidth}
        \hfill \textsc{#1}
    \end{minipage}
    \hfill\vline\hfill
    \begin{minipage}[t]{.80\linewidth}
        \textbf{#2} \\
        \footnotesize{#3}
    \end{minipage}
}

\begin{document}
    \maketitle
    \small
    
    \section{Образование}
        \entry {2019 -- н.в.}
        {Московский физико-технический институт\\
         Физтех-школа прикладной математики и информатики\\
         Информатика и вычислительная техника}
        {Очная форма обучения, бакалавриат, 2 курс}


    \section{Курсы}
        \entry {2019 -- 2020 \\
       %\href {https://github.com/InversionSpaces/problems} {Репозиторий} 
        }
        {Технотрек Mail.ru Group \\
        Курс "Введение в промышленное программирование и структуры данных"}
        {Языки C, C++. В ходе курса я:
        \begin{itemize}
            \item Разработал эмулятор стэковой машины.
            \item Разработал простой компилятор собственного языка высокого уровня для неё.
        \end{itemize} 
        }
          
        \vspace{.2cm}
        
        \entry {2018 \\
       %\href {https://gitlab.informatics.ru/2017-2018/53/HoE} {Репозиторий} 
        }
        {Московская школа программистов при Яндексе \\
        Курс "Промышленное программирование"}
        {Язык Python3. В ходе работы над финальным проектом курса я:
        \begin{itemize}
            \item Был frontend и backend разработчиком маленького видео-хостинга.
            \item Переписал модуль разбора изображения с использованием OpenCV, что ускорило его в несколько раз.
        \end{itemize} 
        }  
        
        \vspace{.2cm}  
    
        \entry {2017 -- 2019 \\
       %\href {https://github.com/Gravifarsh/FreyaCode} {Репозиторий} 
        }
        {Воздушно-инженерная школа Cansat в России\\
        Разработка спускаемого аппарата}
        {Язык C для AVR и STM32. В ежегодных проектах я:
        \begin{itemize}
            \item Был ведущим разработчиком бортового кода.
            \item Был единственным разработчиком бортового кода в год победы моей команды в регулярной лиге.
        \end{itemize} 
        } 
        
        \vspace{.2cm}
        
        \entry {2017 -- 2019 \\
       %\href {} {Репозиторий} 
        }
        {Летняя школа "Слон" в Пущино\\
        Проектная деятельность}
        {Язык C$\boldsymbol{\#}$, Java, C++. В ежегодной школе я:
        \begin{itemize}
            \item Разрабатывал ИИ для MOBA игры (Mail.ru CodeWizards) на C$\boldsymbol{\#}$. 
            \item Разрабатывал клиентскую часть клиент-серверного приложения для Android на Java.
            \item Разрабатывал упрощённую реализацию блокчейна на C$\boldsymbol{++}$.
        \end{itemize} 
        }
        
    \section{Проекты}
        
    \entry {2019 \\
    \href {https://github.com/InversionSpaces/PEGgen} {Репозиторий} }
    {PEGgen - генератор парсеров на рекурсивном спуске \\
    Генерация C++ кода на основе формальной грамматики}
    {Язык Python3. Я реализовал проект пользуясь лециями о рекурсивном спуске моего семинариста.} 
    
    \vspace{.2cm}
    
    \entry {2017 -- 2019 \\
    \href {https://github.com/cansat-rsce/librscs} {Репозиторий} }
    {librscs - библиотека для разработчиков CanSat \\
    Библиотека драйверов перифирии для микроконтроллеров AVR}
    {Язык C для AVR. Я реализовал драйвер модема Iridium9602 и тестировал другие модули.} 

    \section{Навыки}
        \begin{multicols}{2}
            \subsection{Языки программирования}
                \begin{itemize}
                    \item C/C$\boldsymbol{++}$
                    \item Python3
                    \item x86 Assembly
                    \item JavaScript
                    \item Scheme
                \end{itemize}
            \subsection{Языки}
                \begin{itemize}
                    \item Русский - родной
                    \item English - Intermediate
                \end{itemize}
            \subsection{Инструменты разработки}
                \begin{itemize}
                    \item git
                    \item make
                \end{itemize}
        \end{multicols}
        
    \section{Другое}
        \begin{itemize}
            \item Увлекаюсь CTF соревнованиями в формате jeopardy. 
            \item Умею администрировать linux системы на любительском уровне.
        \end{itemize}
    \vspace{\fill}
    \begin{center}
        \large
        \href {https://github.com/InversionSpaces/resume}{Актуальная версия этого резюме}
    \end{center}
\end{document}
